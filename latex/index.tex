En este proyecto se representa un almacén, con sus respectivas salas. Se introducen las clases \char`\"{}\+Almacen\char`\"{}, \char`\"{}\+Sala\char`\"{} y \char`\"{}\+Producto\char`\"{}. Además se usa alguna eterna como el \mbox{\hyperlink{bintree_8hh}{B\+In\+Tree.\+hh}}. Los productos se meten en una ista de productos dentro de la clase almacen para poder hacer el inventario y poder acceder a cada uno de los productos en todo momento. Además la clase almacen llama siempre a la clase sala para las funciones, de modo que la estructura del programa se centra en el almacen y deriva los trabajos a otras clases dependiendo de lo que se tenga que hacer. 