En este proyecto se representa un almacén, con sus respectivas salas. Se introducen las clases \char`\"{}\+Almacen\char`\"{}, \char`\"{}\+Sala\char`\"{} y \char`\"{}\+Stock\char`\"{}. Además se usa alguna externa como el \mbox{\hyperlink{bintree_8hh}{B\+In\+Tree.\+hh}}. La clase stock fuciona de inventario general para poder consultar productos y el propio inventario. Además la clase almacen llama siempre a la clase sala para las funciones, de modo que la estructura del programa se centra en la clase \char`\"{}\+Sala\char`\"{} y \char`\"{}\+Stock\char`\"{} principalmente, pues alli es donde estan implementadas las funciones y las funciones de almacen solo llaman a esas funciones. 